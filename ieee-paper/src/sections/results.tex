\documentclass[conference]{ieee-template}

\begin{document}

\section{Resultados}
Os resultados obtidos na pesquisa são apresentados a seguir. A análise dos dados coletados revela que [insira aqui os principais resultados e descobertas]. 

Os dados foram organizados em tabelas e gráficos para facilitar a visualização. A Tabela \ref{tab:resultados} mostra [descrever o que a tabela representa]. 

\begin{table}[h]
    \centering
    \caption{Resultados da pesquisa}
    \label{tab:resultados}
    \begin{tabular}{|c|c|c|}
        \hline
        Parâmetro & Valor 1 & Valor 2 \\
        \hline
        Exemplo 1 & 10      & 20     \\
        Exemplo 2 & 15      & 25     \\
        \hline
    \end{tabular}
\end{table}

% Além disso, a Figura \ref{fig:exemplo} ilustra [descrever o que a figura representa].

% \begin{figure}[h]
%     \centering
%     \includegraphics[width=0.8\linewidth]{../figures/example-figure.pdf}
%     \caption{Exemplo de figura}
%     \label{fig:exemplo}
% \end{figure}

Os resultados indicam que [discussão sobre os resultados]. A análise estatística realizada confirma que [insira aqui a análise estatística, se aplicável].

\end{document}