\documentclass[conference]{IEEEtran}
\IEEEoverridecommandlockouts
\usepackage{inputenc}[utf8]
\usepackage[brazil]{babel}
\usepackage{cite}
\usepackage{amsmath,amssymb,amsfonts}
\usepackage{algorithmic}
\usepackage{graphicx}
\usepackage{textcomp}
\usepackage{xcolor}
\usepackage{fancyhdr}

\def\BibTeX{{\rm B\kern-.05em{\sc i\kern-.025em b}\kern-.08em
    T\kern-.1667em\lower.7ex\hbox{E}\kern-.125emX}}

\begin{document}
\title{Relatório de Modulação AM}

\author{
\IEEEauthorblockN{Luiz Felipe Rodrigues e Silva}
\IEEEauthorblockA{
\textit{Universidade de Brasília} \\
Brasília-DF, Brasil \\
luizrodriguesesilva@outlook.com.br}
}

\maketitle

\fancyhead{}
\fancyhead[L]{Relatório de Modulação AM}

\begin{abstract}
Este trabalho aborda a modulação em amplitude (AM), uma técnica clássica de transmissão de sinais analógicos, e suas variantes: AM convencional e AM com portadora suprimida (SC). São realizados dois experimentos utilizando o GNU Radio, nos quais são implementados e analisados tanto a modulação quanto a demodulação de sinais em ambas as técnicas. O objetivo é compreender, na prática, os conceitos teóricos estudados sobre modulação AM, observando as diferenças de desempenho e características entre os métodos, bem como os desafios envolvidos no processo de demodulação.
\end{abstract}

\begin{IEEEkeywords}
component, formatting, style, styling, insert.
\end{IEEEkeywords}

\section{Introdução}

A modução é um tecnica que permite a transmissão de sinais analógicos ou digitais através de um meio físico, como o ar ou cabos.
Segundo \cite{b12}, a modulação AM é um processo que envolve a variação da amplitude de uma portadora de alta frequência em função de um sinal modulante, que contém a informação a ser transmitida. Essa técnica é amplamente utilizada em sistemas de comunicação, como rádio e televisão, devido à sua simplicidade e eficácia na transmissão de sinais analógicos.


\subsection{Modulação AM-DSB-SC (Double Sideband Suppressed Carrier)}
A modulação AM-DSB (Double Sideband) é uma forma de modulação em que a portadora e as duas laterais (superior e inferior) são transmitidas. Esse tipo de modulação consiste em multiplicar o sinal de informação por uma portadora, como o sinal mensagem é de banda limitada, isto é, se o sinal $m(t)$ admite transformada de fourier, então $M(f) = 0$ para $|f| > W$, onde $W$ é a largura de banda do sinal.
Como a mensagem tem sua representação espectral, é possivel deslocar a mensagem para uma nova frequência utilizando a propriedade de modulação da transformada de Fourier, que nos diz que a multiplicação no domínio do tempo por uma exponencial complexa resulta em um deslocamento espectral no domínio da frequência. Multplicando o sinal de informação $m(t)$ por uma portadora $c(t) = A_{c} \cos(2 \pi f_{c} t)$, temos:
\begin{equation}
    s(t) = m(t) c(t) = A_{c} m(t) \cos(2 \pi f_{c} t)
\end{equation}

onde $A_{c}$ é a amplitude da portadora e $f_{c}$ é a frequência da portadora. Podemos expandir a expressão acima utilizando a identidade trigonométrica $\cos(x) = \frac{e^{jx} + e^{-jx}}{2}$, resultando em:

\begin{equation}
    s(t) = \frac{A_{c}}{2} m(t) e^{j 2 \pi f_{c} t} + \frac{A_{c}}{2} m(t) e^{-j 2 \pi f_{c} t}
\end{equation}

pode-se aplicar a transformada de Fourier em ambos os lados da equação, resultando em:

\begin{equation}
    S(f) = \frac{A_{c}}{2} (M(f - f_{c}) + M(f + f_{c}))
\end{equation}
Por exemplo, para uma mensagem $\cos(2\pi f_{m} t)$ com frequência $f_{m} = 100\,\text{Hz}$, uma portadora $\cos(2\pi f_{c} t)$ de $f_{c} = 1\,\text{kHz}$ com amplitude $A_{c} = 1$, realizando a modulação AM-DSB, os espectros para a mensagem, portadora e sinal modulado são mostrados nas Figuras~\ref{fig:espectro_mensagem}, \ref{fig:espectro_portadora} e \ref{fig:espectro_modulado}, respectivamente. O espectro do sinal modulado AM-DSB apresenta duas bandas laterais, uma superior e outra inferior, que contêm a mesma informação, além da portadora.

\begin{figure}[h]
    \centering
    \includegraphics[width=0.5\textwidth]{images/FFT (m(t))_full.png}
    \caption{Espectro do sinal de mensagem. Fonte: Autor.}
    \label{fig:espectro_mensagem}
    \centering
\end{figure}

\begin{figure}[h]
    \centering
    \includegraphics[width=0.5\textwidth]{images/FFT (c(t))_full.png}
    \caption{Espectro da portadora. Fonte: Autor.}
    \label{fig:espectro_portadora}
    \centering
\end{figure}

\begin{figure}[h]
    \centering
    \includegraphics[width=0.5\textwidth]{images/FFT (s(t))_full.png}
    \caption{Espectro do sinal modulado AM-DSB. Fonte: Autor.}
    \label{fig:espectro_modulado}
    \centering
\end{figure}


O diagrama de blocos da modulação AM-DSB é apresentado na Figura \ref{fig:modulacao_am}, onde o sinal de informação $m(t)$ é multiplicado pela portadora $c(t)$, resultando no sinal modulado $s(t)$. A demodulação do sinal AM-DSB pode ser realizada utilizando um detector de envoltória, que recupera o sinal de informação original a partir do sinal modulado.

\begin{figure}[h]
    \centering
    \includegraphics[width=0.5\textwidth]{images/modulacao_am.png}
    \caption{Diagrama de blocos da modulação AM-DSB. Fonte: Autor.}
    \label{fig:modulacao_am}
\end{figure}


Podemos demostrar os resultados a partir da transformada de Fourier, onde a transformada de Fourier do sinal modulado $m(t)$ é dada por:

\begin{equation}
    M(f) = \frac{1}{2} \left( \delta (f - 100) + \delta (f + 100) \right)
\end{equation}

substituindo na equação (3), temos:

\begin{align}
    S(f) = \frac{1}{4} \big(
        &\ \delta (f - 100 - 1000) + \delta (f + 100 - 1000) \notag \\
        &+ \delta (f - 100 + 1000) + \delta (f + 100 + 1000)
    \big)
\end{align}

Conforme mostrado no espectro \ref{fig:espectro_modulado}, a portadora é suprimida.


\subsection{demodulação AM-DSB-SC (Double Sideband Suppressed Carrier)}

A demodulação AM-DSB-SC é um processo que visa recuperar o sinal de informação original a partir do sinal modulado. O método mais comum para realizar essa demodulação é o uso de um multiplicador, que multiplica o sinal modulado por uma cópia da portadora. Esse processo resulta em um sinal que contém a informação original, mas também inclui uma componente de alta frequência que deve ser filtrada.

\begin{equation}
    s(t) = m(t) c(t) = A_{c} m(t) \cos(2 \pi f_{c} t)
\end{equation}

\begin{equation}
    y(t) = s(t) c(t) \cos(2 \pi f_{c}t)= A_{c} m(t) \cos(2 \pi f_{c} t)^{2}
\end{equation}

Linarizando o coseno, tempos:

\begin{equation}
    r(t) = \frac{A_{c}}{2} m(t) + \frac{A_{c}}{2} m(t) \cos(4 \pi f_{c} t)
\end{equation}

A transformada de Fourier do sinal demodulado $r(t)$ é dada por:

\begin{equation}
    Y(f) = \frac{A_{c}}{2} M(f) + \frac{A_{c}}{2} M(f - 2 f_{c}) + \frac{A_{c}}{2} M(f + 2 f_{c})
\end{equation}

Aplicando um filtro passa-baixa com largura de banda $W$ para eliminar a componente de alta frequência, obtemos o sinal de informação original:

\begin{equation}
    Y_{LPF}(f) = \frac{A_{c}}{2} M(f)
\end{equation}

O sinal é recuperado com uma amplitude reduzida, o que pode ser compensado por um amplificador. O diagrama de blocos da demodulação AM-DSB-SC é apresentado na Figura \ref{fig:demodulacao_am}, onde o sinal modulado $s(t)$ é multiplicado pela portadora $c(t)$, resultando no sinal demodulado $r(t)$. Em seguida, um filtro passa-baixa é aplicado para recuperar o sinal de informação original.

\begin{figure}[h]
    \centering
    \includegraphics[width=0.5\textwidth]{images/demodulacao_am.png}
    \caption{Diagrama de blocos da demodulação AM-DSB-SC. Fonte: Autor.}
    \label{fig:demodulacao_am}
\end{figure}

Um dos principais desafios da modulação em amplitude com portadora suprimida (DSB-SC) é a necessidade de sincronização de fase entre o sinal modulado e a portadora na demodulação. Caso haja um desvio de fase entre a portadora original e a gerada no receptor, o sinal demodulado apresentará distorções significativas. Para mitigar esse problema, uma abordagem comum é transmitir um tom piloto juntamente com o sinal modulado. Esse tom é uma pequena fração da portadora original, inserida com baixa amplitude, e pode ser isolado no receptor por meio de um filtro de banda estreita. No entanto, a presença do tom piloto implica que a portadora não está totalmente suprimida, o que descaracteriza a modulação como DSB-SC pura.

Outra alternativa mais robusta é o uso de um PLL (Phase-Locked Loop), um circuito que sincroniza automaticamente a fase da portadora local com a fase do sinal modulado recebido. O PLL ajusta continuamente a frequência e a fase do oscilador local, permitindo uma demodulação mais precisa mesmo na presença de ruídos e desvios de fase. A Figura \ref{fig:demodulacao_am_pll} ilustra o esquema de demodulação utilizando PLL.

\begin{figure}[h]
    \centering
    \includegraphics[width=0.5\textwidth]{images/demodulacao_am_pll.png}
    \caption{Diagrama de blocos da demodulação AM-DSB-SC com PLL. Fonte: Autor.}
    \label{fig:demodulacao_am_pll}
    \centering
\end{figure}





\section{Metodologia}

\subsection{Modulação AM-DSB-SC (Double Sideband Suppressed Carrier)}

Neste experimento foi utilizado o GNU Radio Companion (GRC) para implementar a modulação AM-DSB-SC. O sinal de mensagem foi gerado utilizando um bloco de fonte de sinal senoidal, com frequência fm de 1kHz e amplitude de 1. A portadora foi gerada com uma frequência de 5kHz e amplitude de 1. O sinal modulado foi obtido multiplicando o sinal de mensagem pela portadora.

O diagrama de blocos da modulação AM-DSB-SC no GNU Radui é apresentado na Figura \ref{fig:modulacao_am_sc}
\begin{figure}
    \centering
    \includegraphics[width=0.5\textwidth]{images/modulacao_gnu_am_dsb_sc.png}
    \caption{Diagrama de blocos da modulação AM-DSB-SC. Fonte: Autor.}
    \label{fig:modulacao_am_sc}
\end{figure}

Os seguintes blocos foram utilizados:

\begin{itemize}
    \item \textbf{Signal Source}: Gera o sinal de mensagem com frequência de 1000 Hz e amplitude de 1 .
    \item \textbf{Signal Source}: Gera a portadora com frequência de 0 à 20kHz,e amplitude de 1 V.
    \item \textbf{Signal Source}: Gera o oscilador local com frequência fixa de 5kHz.
    \item \textbf{QT GUI Range}: Permite ajustar a frequência da portadora. Aqui foi denido entre 0 e 20kHz.
    \item \textbf{Multiply}: Multiplica o sinal de mensagem pela portadora, resultando no sinal modulado AM-DSB-SC.
    \item \textbf{Virtual Sink}: Permite armazenar o sinais e usalos no bloco Virtual Source.
    \item \textbf{Virtual Source}: Permite usar os sinais armazenados no bloco Virtual Sink.
    \item \textbf{QT GUI Time Sink}: Exibe o sinal modulado no domínio do tempo.
    \item \textbf{QT GUI Frequency Sink}: Exibe o espectro do sinal modulado no domínio da frequência.
\end{itemize}


Para mostrar a perda da mensagem pela falta de sincronismo, foi ajustado a frequência da portadora para uma frequência de 10Khz.

\subsection{Demodulação AM-DSB-SC (Double Sideband Suppressed Carrier)}

De maneira semelhante, foi implementada a demodulação do sinal AM-DSB-SC utilizando o GNU Radio Companion. O sinal modulado foi multiplicado novamente por um sinal da portadora (oscilador local) com a mesma frequência e fase da portadora original, seguido de um filtro passa-baixas para recuperar o sinal de mensagem.

O diagrama de blocos da demodulação AM-DSB-SC no GNU Radio é apresentado na Figura \ref{fig:demodulacao_am_sc_gnu}.

\begin{figure}
    \centering
    \includegraphics[width=0.5\textwidth]{images/demodulacao_gnu_am_dscb_sc.png}
    \caption{Diagrama de blocos da demodulação AM-DSB-SC. Fonte: Autor.}
    \label{fig:demodulacao_am_sc_gnu}
\end{figure}

Os seguintes blocos foram utilizados:

\begin{itemize}
    \item \textbf{Virtual Source}: Recebe o sinal modulado armazenado anteriormente.
    \item \textbf{Signal Source}: Gera o oscilador local com frequência igual à portadora (5kHz).
    \item \textbf{Multiply}: Multiplica o sinal modulado pelo oscilador local.
    \item \textbf{Throtle}: Controla a taxa de amostragem do fluxo de dados, evitando sobrecarga do processamento durante a simulação.
    \item \textbf{Low Pass Filter}: Frequencia de corte em 2kHz, com ganho de 2 e a janela na forma retangular.
    \item \textbf{QT GUI Time Sink}: Exibe o sinal demodulado no domínio do tempo.
    \item \textbf{QT GUI Frequency Sink}: Exibe o espectro do sinal demodulado no domínio da frequência.
\end{itemize}

Para demonstrar a importância do sincronismo, foi realizada a demodulação com o oscilador local defasado ou com frequência diferente, observando a distorção ou perda do sinal de mensagem.


\input{sections/experiments}
\section{Resultados}

Para a demodulação do sinal FM, foi obtido o seguinte gráfico:

\begin{figure}[!h]
    \centering
    \includegraphics[width=0.5\textwidth]{images/sinal_fm_p.png}
    \caption{Gráfico do sinal demodulado.}
    \label{fig:resultado_demodulacao}
\end{figure}

aumentado a amplitude do sinal mensagem, tempo mais armonicas sendo geras.

Na demodulação do sinal FM, foi observado que o aumento da amplitude do sinal mensagem resultou em um maior número de harmônicas sendo geradas. Isso é esperado, pois a modulação em frequência (FM) é sensível à amplitude do sinal de mensagem, e um aumento na amplitude leva a uma maior variação na frequência da portadora, resultando em uma maior complexidade espectral do sinal demodulado.

\begin{figure}[!h]
    \centering
    \includegraphics[width=0.5\textwidth]{images/cdm.png}
    \caption{Gráfico do sinal demodulado.}
    \label{fig:resultado_demodulacao}

\end{figure}


\section{Discussão e Conclusão}

O experimento demonstrou com sucesso a modulação e demodulação de um sinal FM utilizando o GNU Radio. A geração do sinal FM foi realizada de forma eficiente, com a utilização de um VCO e ajustes dinâmicos na amplitude do sinal de mensagem. A demodulação foi efetiva, permitindo a recuperação do sinal original com precisão.
\begin{thebibliography}{00}
\bibitem{b12} NETO, Vicente S. Sistemas de Comunicação - Serviços, Modulação e Meios de Transmissão. Rio de Janeiro: Érica, 2015. E-book. p.44. ISBN 9788536522098. Disponível em: https://integrada.minhabiblioteca.com.br/reader/books/9788536522098/. Acesso em: 17 mai. 2025.
\bibitem{b13} Proakis, J. G., \& Salehi, M. (2005). Fundamentals of Communication Systems. McGraw-Hill.
\end{thebibliography}

\vspace{12pt}
\end{document}